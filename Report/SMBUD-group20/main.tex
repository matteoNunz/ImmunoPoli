\documentclass[a4paper, 12p]{article}
\usepackage{graphicx}
\usepackage{hyperref}
\usepackage{enumitem}
\usepackage{csquotes}
\usepackage[T1]{fontenc}
\usepackage{caption}
\usepackage{booktabs}
\usepackage{xcolor}
\usepackage{float}
\usepackage{longtable}
\usepackage{tasks}
\usepackage{tabularx}
\usepackage{titlesec}
\usepackage{subcaption}
\newcommand{\link}[1]{{\color{blue}\href{#1}{#1}}}

\titleformat{\paragraph}
{\normalfont\normalsize\bfseries}{\theparagraph}{1em}{}
\titlespacing*{\paragraph}
{0pt}{3.25ex plus 1ex minus .2ex}{1.5ex plus .2ex}
\setlength {\marginparwidth }{2cm}
\begin{document}
\begin{titlepage}
    \centering
    {\scshape\large AY 2021/2022 \par}
    \vfill
    \includegraphics[width=100pt]{images/logo-polimi-new.pdf}\par\vspace{1cm}
    {\scshape\LARGE Politecnico di Milano \par}
    \vspace{1.5cm}
    {\huge\bfseries SMBUD 2021 - Project work\@ \\ delivery 1  \par}
    
    \vspace{1.5cm}
 
     {\huge\bfseries ImmunoPoli \par}

    \vspace{1.5cm}

    {\large \textbf{Group 20} \par
    
    
        
    \begin{center}
        {\begin{tabular}{l l l}
        Margherita & Musumeci & (10600069)\\
        Matteo & Nunziante & (10670132)\\
        Piero & Rendina & (10629696)\\
        Andrea & Sanchini & (10675541)\\
        Enrico & Zuccolotto & (10666354)\\
        \end{tabular}}
        
    \end{center}
    }
    \vfill
    \begin{center}
        {\large \textbf{Professor}\par
            Marco Brambilla}
          \vspace{1cm}
        {\large \\ \today \par}
    \end{center}
\end{titlepage}
\hypersetup{%
    pdfborder = {0 0 0}
}
\thispagestyle{plain}

\mbox{}


\tableofcontents
\pagenumbering{gobble}

\newpage

\setcounter{page}{1}
\pagenumbering{arabic}
\section{Introduction}
At the beginning of 2020, the COVID-19 pandemic arose. It entailed many restrictions, lockdowns, and freedom limitations all around the world.
Keeping track of all possible contacts among people is a crucial step to help governments and countries address the worldwide viral spreading.
Furthermore, collecting and understanding data is the key to have a clear snapshot of all the people currently dealing with the virus, including also vaccines efficacy and contagion rate.
The project aims to provide an efficient database for storing all information about people and an application to query it and perform additional analysis.
The database is based on a graph data structure and makes usage of the NoSQL paradigm.
It stores data coming from various sources, such as data collected either through inspecting public building access or data provided by contact tracing applications in addition to basic personal information about people.

\section{Assumptions}
\label{section 2}
The project relies on the hypotheses described below:

\begin{enumerate}
    \item All information provided by people is truthful.
    \item \label{assumption_2} No one visits a place knowing that their last test is positive. 
    \item The limit for the number of vaccine doses injectable is \emph{two}.
    \item There is no house inhabited by underage people only.
    \item The people are prevented from changing their residence address.
    \item The period ranging from virus incubation to test outcome acknowledgement is \emph{seven days}. It entails that the people who had contact with the positive person, up to seven days before the test, have been potentially exposed to the virus. To be declared safe, each user involved in the exposures needs a negative swab at least seven days after the contact.
    \item Since the \textit{covid exposure} relationship represents a possible contact with the virus, for each user, the actual contagion refers to the last positive swab. 
    \item Every address is associated with just one house or public place.
    \item The database includes only entries related to indoor places. Those are assumed to be accessible through a covid certificate vaccination scan. The registration before entering is feasible also by giving own email. Moreover, for those not equipped with the certificate, access is allowed by providing their references. 
\end{enumerate}    
    


\newpage

\section{Conceptual model of the database}
\begin{figure}[h]
    \noindent\makebox[\textwidth]{
    \includegraphics[width=0.8\paperwidth]{images/ImmunoPoli-ER.jpg}
    }
\caption{\textit{Entity Relationship model}}
\end{figure}

\subsection{Entities}
\begin{itemize}
    \item \textbf{Person:} man or woman identified by a unique ID. The ID is also the key for logging into the app. The property \textit{app} is a boolean attribute used to identify whether or not the person has any type of contact tracing application.
    \item \textbf{Place:} general concept of place characterized by an address.
    \item \textbf{Private house:} place that hosts either a group of flatmates or a family. The name of the house corresponds to the owner's name and surname. 
    \item \textbf{Public place:} represents indoor places accessible providing personal information for traceability purposes.
    \item \textbf{Covid Vaccination:} the COVID-19 vaccines. 
    They are identified through their company name and the production plant from where they come.
    \item \textbf{Test:} the COVID-19 tests. They are identified through their class. The distinction is between the tests that detect the viral presence through the \textit{PCR}, \textit{Molecular}, \textit{Antibody} (to check the presence of the virus antagonists) and in the end, the \textit{Rapid} ones. 
\end{itemize}

\subsection{Relationships}
\begin{itemize}
    \item The \textit{\textbf{covid exposure}} relationship represents a \emph{possible viral infection}. Once a positive test is inserted into the database, the system triggers a command\footnote{see section \ref{section: 6} for further details about the command.} to easily find all the people at risk of contagion and to add an instance of the relationship for all of them. People are obliged to make a swab, and in case of a negative result after at least seven days, the relationship is deleted.
    \item The \textit{\textbf{live}} relationship binds a person with where he resides.
    \item The \textit{\textbf{visit}} relationship involves all the public places where it is more likely that personal information is collected. It serves both realism and simplicity purposes because we don't need to take care of outdoor spots.
    \item The \textit{\textbf{app contact}} relationship gathers all contacts detected by any tracking system that supports localization and applications installation, regardless of its type.
    \item The \textit{\textbf{make test}} relationship links the user with a test he has made. It contains information such as the hour, the date and the outcome.
    \item The \textit{\textbf{get vaccine}} relationship connects the user to the vaccine he got. As well as \emph{make test}, it is characterized by the date and hour of the injection.
\end{itemize}
\newpage
\section{Dataset description}
\label{section: 4}
The dataset is drawn from a random generator. It allows enforcing parameters such as the number of visits, tests, covid vaccinations, families and the probability of being positive. \\
The generator sequentially builds the graph starting from people, places, families and groups of flatmates. \\
Families are sets of people with at most two different surnames, whereas flatmates are combined randomly. \\
Afterwards, it adds all the vaccinations ensuring that the first dose of vaccine grants a one-month validity certification. The second vaccination provides one-year validity certification employing the same vaccine as the first one. \\
Then, it builds instances of \textit{make test} relationships. 
Every time a positivity status is confirmed, we check the previous contacts and the visited places to look for any \textit{covid exposure} relationships. Moreover,  to ensure the assumption in section 2.\ref{assumption_2}, those who have been tested positive are prevented from going out from then on.

\section{Queries}
\begin{enumerate}[leftmargin=*,label=\textbf{\thesection.\arabic*}]
\item Find undirected contacts stored in the \textit{app contact} relationship.
\begin{figure}[h]
    \centering
    \includegraphics[width=\textwidth]{images/find_indirected_app_contacts.png}
\end{figure} 
\item Find the mean age of the infected\footnote{We refer to infected people as the positive-tested in the last ten days.} people in the last ten days.
\begin{figure}[h]
    \centering
    \includegraphics[width=\textwidth]{images/mea_age_last_days.png}
\end{figure} 
\item Find people that live with someone who has been tested positive in the last ten days.
\begin{figure}[h]
    \centering
    \includegraphics[width=\textwidth]{images/mates_of_positive.png}
\end{figure} 
\item Find the homes where resides someone tested positive in the last ten days.
\item Find the number of test performed in the last month.
    \item Find the number of infected people in a given month.
    \item Find the number of vaccinations performed in a given month.
    \item Find the number of people with a single vaccine dose in a given CAP.
    \item Find the number of people vaccinated in a given CAP.
    \item Find the first five places according to the rate of positive people in the last month.
    \item Find the people exposed to the virus who have not been tested yet.\footnote{They haven't made a swab from the date of the last covid exposure until now.} 
    \item Find the percentage of infected people that got at least one dose of vaccine.
\end{enumerate}
\newpage
\section{Commands}
\label{section: 6}
As already seen in section \ref{section: 4}, the generator exploits several basic commands to build the graph database. 
Concerning the application, it triggers neo4j library routines to manage the database changes and to guarantee its consistency. 
In this section, we report the behaviour of the most relevant commands and explore their employment within the Python code. \\*
Below, there is the list of commands: \\*
\begin{itemize}[leftmargin=*]
    \item \textbf{Creation of COVID exposures command.} It is triggered whenever the government employee inserts a positive test. At first, it collects all people IDs. Then, it builds the relationships according to the IDs found.
    \begin{figure}[h]
        \includegraphics[width=\textwidth]{images/create_covid_exposure_command/family_contacts.png}
        \captionsetup{skip=7pt}
        \caption{\textit{Query that retrieves either flatmates' or relatives' IDs of a positive person.}}
    \end{figure}
    \begin{figure}[h]
        \includegraphics[width=\textwidth]{images/create_covid_exposure_command/app_contacts.png}
        \captionsetup{skip=7pt}
        \caption{\emph{Query that retrieves app contacts IDs for a positive person.}}
    \end{figure}
    \begin{figure}[ht]
        \includegraphics[width=\textwidth]{images/create_covid_exposure_command/visit_contacts.png}
        \captionsetup{skip=7pt}
        \caption{\textit{Query that retrieves people met by a positive person in a certain place.}}
    \end{figure}
    \begin{figure}[!htb]
        \includegraphics[width=\textwidth]{images/create_covid_exposure_command/exposed_people.png}
        \captionsetup{skip=7pt}
        \caption{\textit{Command run over all previously collected IDs to instantiate the exposures.}}
    \end{figure}
    \newpage
    \item \textbf{Deletion of possible exposures command.} It is triggered whenever the government employee inserts a negative test. It looks for the exposures that may have involved the negative tested person and, it deletes all of them.
    \begin{figure}[h]
        \includegraphics[width=\textwidth]{images/commands/delete_exposures.png}
        \captionsetup{skip=7pt}
        \caption{\textit{Deletion command. Note that the search is done by ID.}}
    \end{figure}
    
    \item \textbf{Update of personal information command}. Differently from the others above, this command is called when a generic user is modifying his information. It supports email and phone number changes.
    \begin{figure}[h]
        \includegraphics[scale = 0.75]{images/commands/info_update.png}
        \captionsetup{skip=7pt}
        \caption{\textit{Update of telephone number and email provided the person ID.}}
    \end{figure}
    
    \item \textbf{Creation of a new test command.} It adds the given \emph{date, hour and result} as attributes of the relationship.
    \begin{figure}[h]
        \includegraphics[width=\textwidth]{images/commands/add_new_test.png}
    \end{figure}
    \item \textbf{Creation of a new visit instance command.} It is activated whenever a location manager adds data related to a \emph{visit}. The \emph{date} is set by default to the current one, whereas the user has to specify the \emph{hour} as attribute of the relationship.
    \begin{figure}[!h]
        \includegraphics[width = \textwidth]{images/commands/visit_creation.png}
    \end{figure}
\end{itemize}

\newpage
\section{Application}
The application is the database entry point, and it is projected to be run either by a a government employee or a generic user. The starting page shows the two possible ways to access it.
\begin{figure}[h]
    \centering
    \begin{minipage}{0.475\textwidth}
        \centering
        \includegraphics[width=\textwidth]{images/starting_page.png} 
        \caption{\textit{starting page}}
        \label{figure 2}
    \end{minipage}\hfill
    \begin{minipage}{0.475\textwidth}
        \centering
        \includegraphics[width=\textwidth]{images/login_page.png} 
        \caption{\textit{login page}}
        \label{figure 3}
    \end{minipage}
\end{figure}
\newline
The first category of users has access to two features:
\begin{itemize}
    \item changing his/her personal information;
    \item examining the places he has visited and the risk he linked to them;
    \item checking his covid exposures;
    \item seeing the history of his tests and related outcomes.
\end{itemize}
The second category instead, has full access to the features:
\begin{itemize}
    \item adding information concerning COVID tests;
    \item monitoring COVID trends;
\end{itemize}

\newpage
\section{Resources}
To access the application the IDs are needed. Here there is the list of the IDs to sign in.
\begin{center}
        {\begin{tabular}{|c|c|}
        \hline
        User/App Manager & Location Manager \\*
        \hline
        4027 & 4590 \\*
        4058 & 4560 \\*
        4084 & 4551 \\*
        4307 & 4623 \\*
        3951 & 4592 \\*
        4045 & 4588 \\*
        \hline
        \end{tabular}}
\end{center}

\noindent
To add new tests is required one of the following ID.
\begin{center}
    {\begin{tabular}{|c|c|}
    \hline
    Test Name & Test ID \\*
    \hline
    Rapid & 4653 \\*
    Molecular & 4654 \\*
    PCR & 4655 \\*
    Antibody & 4656 \\*
    \hline
    \end{tabular}}
\end{center}

\noindent
All the contents covered in this report are available at the link below:\\* \link{https://github.com/matteoNunz/ImmunoPoli}
\end{document}