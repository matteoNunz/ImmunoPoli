\documentclass[a4paper, 12p]{article}

\makeindex


\title{\Large{\textbf{SAMBUD Project}}}
\author{by Group 20}
\date{14 November 2021}


\begin{document}
\maketitle

\section{Introduction}
At the beginning of 2020, the COVID-19 pandemic arose. It entailed many restrictions, lockdowns, and freedom limitations all around the world.
Keeping track of all possible contacts among people is a crucial step to help governments and countries address the worldwide viral spreading.
Furthermore, collecting and understanding data is the key to have a clear snapshot of all the people currently dealing with the virus, including also vaccines efficacy and contagion rate.
The project aims to provide an efficient database for storing all information about people and an application to query it and perform additional analysis.
The database is based on a graph data structure and makes usage of the NoSQL paradigm.
It stores data coming from various sources such as data collected through inspecting public building access, or data provided by contact tracing applications in addition to basic personal information about people.
\\* The application is the database entry point, and it is projected to be run either by a generic user or a government employee.
\\* The first category of users has access to two features:
\begin{itemize}
    \item adding information concerning COVID tests.
    \item getting notifications whenever he gets in touch with someone who has been tested positive.
\end{itemize}
The second category instead, has full access to the features.

\section{Assumptions}
The project relies on the hypotheses described below:

\begin{enumerate}
    \item All information provided by people is truthful.
    \item The property \textit{app} in Person entity is a boolean attribute used to identify whether or not the person itself has a contact tracing application.
    \item The period ranging from virus incubation to test outcome acknowledgment is \textbf{seven days}.
    \item Every address is associated with just one family or group of flatmates.
\end{enumerate}


\end{document}